\newcommand{\doctitle}{计算机系统结构第四次作业}
\documentclass[oneside,a4paper]{article}

\usepackage{parskip}
\usepackage{subfigure}
% \usepackage{geometry}
\usepackage{amsmath}
\usepackage[dvipdfm]{graphicx} 
\usepackage{amsthm,amssymb}
\usepackage{tikz}
\usepackage{fontspec,xltxtra,xunicode}


\usepackage{fancyvrb}
% \usepackage{fancybox}
\usepackage{listings}
\lstset{numbers=left, 
  numberstyle=\scriptsize,
  frame=single,
  flexiblecolumns=false,
  language=,
  texcl=true,
  escapechar=\%,
  basicstyle=\ttfamily\small, 
  breaklines=true,
  extendedchars=true,
  showstringspaces=false,
  keywordstyle=\bfseries}

% \usepackage{indentfirst} 


\usepackage{pstricks} 
\usepackage[dvipdfm,a4paper,
hyperindex=true,
backref=section,
bookmarks=true,
bookmarksnumbered=true,
pdfpagemode=UseOutlines,
pdffitwindow=true,
linkbordercolor=white, % 链接边框设置为白色
citebordercolor=white, % 链接边框设置为白色
urlbordercolor=white]{hyperref}
          

% \DefineShortVerb{\|}

\usepackage{xeCJK}
\setCJKmainfont[BoldFont={Adobe Heiti Std}, ItalicFont={Adobe Kaiti Std}]{Adobe Song Std}
\setCJKmonofont{Adobe Fangsong Std}
\xeCJKsetcharclass{"0}{"2E7F}{0}
\xeCJKsetcharclass{"2E80}{"FFFF}{1}

\setmainfont[Mapping=tex-text]{Linux Libertine O}
\setsansfont[Mapping=tex-text]{Linux Biolinum O} 
\setmonofont{Courier 10 Pitch} 
\punctstyle{quanjiao} 


\renewcommand{\figurename}{图}
\renewcommand{\tablename}{表}
\renewcommand{\lstlistingname}{程序清单}

\newenvironment{solve}{%
  \settowidth{\leftskip}{\textit{解:}\ }%
  \makebox[0pt][r]{\textit{解:}\ }%
  \ignorespaces}{\qed\par\ignorespacesafterend}

\author{\textsc{Computer Science} 0813\\谢松 U200814454}
\title{\doctitle}
\hypersetup{
  pdftitle={\doctitle},
  pdfauthor={谢松},
  pdfsubject={\doctitle}}
\usepackage{booktabs}

\begin{document}


\maketitle

%%% Local Variables: 
%%% mode: latex
%%% TeX-master: t
%%% End: 


\section{Ex 7.9}

\begin{solve}
  第一级cache不命中率
  \[M_1 = \frac{110}{3000} \approx 3.7 \% ;\]

  第二级cache不命中率
  \[M_2 = \frac{55}{110} = 50 \% ;\]

  全局cache不命中率
  \[M_{\mathrm{global}} = M_1 \times M_2 \approx 1.83 \% \]
\end{solve}

\section{Ex 7.10}

\begin{solve}
  记时钟周期为$T_C$, 不命中开销为$T_{\mathrm{miss}}$, 程序指令数$IC$, 每条指令平均访
  存数为$C_{\mathrm{Mem}}$, 不命中率为$r_{\mathrm{miss}}$

  \paragraph{直接映射}
  
  平均访问时间
  \begin{align*}
    T_{\mathrm{access}} &= 1C + r_{\mathrm{miss}}T_{\mathrm{miss}} \\
    &= 2 + 80 \times 0.014\\
    &= 3.12 \mathrm{ns}
  \end{align*}

  CPU 时间
  \begin{align*}
    T_{\mathrm{CPU}} &= IC \times (CPI + C_{\mathrm{Mem}}
    r_{\mathrm{miss}}
    T_{\mathrm{miss}}) \times T_C\\
    &= 2IC(2 + 1.2 \times 0.014 \times 80)\\
    &= 6.688IC
  \end{align*}
  
  \paragraph{2路组相联}
  平均访问时间
  \begin{align*}
    T_{\mathrm{access}}' &= 1C + r_{\mathrm{miss}}'T_{\mathrm{miss}} \\
    &= 2 + 80 \times 0.010\\
    &= 2.8 \mathrm{ns}
  \end{align*}

  CPU 时间
  \begin{align*}
    T_{\mathrm{CPU}}' &= IC \times (CPI + C_{\mathrm{Mem}}
    r_{\mathrm{miss}}'
    T_{\mathrm{miss}}) \times T_C'\\
    &= 2 \times 1.1 \times IC(2 + 1.2 \times 0.010 \times 80)\\
    &= 6.512IC
  \end{align*}
\end{solve}

\section{Ex 7.11}

\begin{solve}
  \begin{enumerate}
  \item 平均访存时间
    \begin{equation}
      T_{\mathrm{access}} = T_{\mathrm{hit}} + T_{\mathrm{pseudo}}r_{\mathrm{pseudo}}
      + T_{\mathrm{miss}}r_{\mathrm{miss}}
    \end{equation}
    其中, $r_{\mathrm{miss}}$为伪相联的不命中率, 即2路组相联的不命中率;
    $T_{\mathrm{miss}}$ 为伪相联的不命中开销; $r_{\mathrm{pseudo}}$ 为
    伪命中率,
    \begin{align}
      r_{\mathrm{pseudo}} &= r_{\mathrm{2hit}} - r_{\mathrm{1hit}} \\
      &= r_{\mathrm{1miss}} - r_{\mathrm{2miss}}
    \end{align}
  \item 根据题目条件, 代入数据, 得到对2KB cache, 平均访存时间
    \begin{align*}
      T_{\mathrm{access}} = 4.822;
    \end{align*}
    对128KB cache, 平均访存时间
    \begin{align*}
      T_{\mathrm{access}} = 1.353.
    \end{align*}

  \end{enumerate}
\end{solve}


\section{Ex 7.12}

\begin{solve}
  记存储器带来的延迟为$t_m$, 则有实
  际CPI
  \begin{equation}
    CPI = CPI_{\mathrm{exec}} + r_{\mathrm{miss}}t_m
  \end{equation}
  其中, $r_{\mathrm{miss}}$ 为发生cache不命中的概率.
  
  记取指停顿$t_I$, 取数据停顿$t_D$,
  TLB停顿$t_{\mathrm{TLB}}$. 则有
  \begin{equation}
    t_m = t_I + r_{\mathrm{TLB}}t_{\mathrm{TLB}} + r_D(t_D  + r_{\mathrm{TLB}}t_{\mathrm{TLB}})
  \end{equation}
  其中, $r_D$为数据传送指令的比例.

  记主存延迟为$t_0$, 取指操作为读内存操作, 访问数据操作可能读, 可能写.
  当发生miss, 且miss的块为脏块时, 需要先将原先的块写回主存, 再调入新的
  块. 故
  \begin{equation}
    t_I = t_D = t_0 + p + r_dp
  \end{equation}
  其中, $r_d$为脏块比例, $p$为从主存中传送一个块的时间, 且
  \begin{equation}
    p = \frac{32}{4} = 8
  \end{equation}

  由上述公式, 可得
  \begin{equation}
    CPI = CPI_{exec} + r_{\mathrm{miss}} (1+r_D)(t_0 + (1+r_d)p + r_{\mathrm{TLB}}t_{\mathrm{TLB}})
  \end{equation}


  \begin{enumerate}
  \item 在理想TLB情况下, TLB失效的开销为0.

    对16KB直接映像cache, 有
    \begin{align*}
      CPI_1 &= 1.5 + 0.029 (1+0.2) (40 + 8(1 + 0.5))\\
      &= 3.3096
    \end{align*}

    对16KB 2路组相联混合cache, 有
    \begin{align*}
      CPI_2 &= 1.5 + 0.022 (1+0.2) (40 + 8(1 + 0.5))\\
      &= 2.8728
    \end{align*}

    对32KB 直接映像相联混合cache, 有
    \begin{align*}
      CPI_3 &= 1.5 + 0.020 (1+0.2) (40 + 8(1 + 0.5))\\
      &= 2.748
    \end{align*}
  \item 在实际TLB情况下, TLB失效的开销为20时钟周期.

    对16KB直接映像cache, 有
    \begin{align*}
      CPI_1' &= 1.5 + 0.029 ((1+0.2) (40 + 8(1 + 0.5) + 0.002 \times 20)\\
      &= 3.11
    \end{align*}

    对16KB 2路组相联混合cache, 有
    \begin{align*}
      CPI_2' &= 1.5 + 0.022 (1+0.2) (40 + 8(1 + 0.5) + 0.002 \times 20)\\
      &= 2.874
    \end{align*}

    对32KB 直接映像相联混合cache, 有
    \begin{align*}
      CPI_3' &= 1.5 + 0.020 (1+0.2) (40 + 8(1 + 0.5) + 0.002 \times 20)\\
      &= 2.748
    \end{align*}
\end{enumerate}
\end{solve}

\section{Ex 7.14}

\begin{solve}
  \begin{enumerate}
  \item 对写直达cache, cache在任何时候都不应该有被修改过的块(即dirty
    block), 这与题目给出的条件冲突. 在此, 按没有dirty block计算.

    读命中
    \begin{align*}
      r_{\mathrm{hit}} &= 0.95 \times 0.75 \times 0\\
      &= 0
    \end{align*}


    读不命中 
    \begin{align*}
      r_{\mathrm{miss}} &= 0.05 \times 0.75 \times 2\\
      &= 0.075
    \end{align*}
    
    写命中 
    \begin{align*}
      w_{\mathrm{hit}} &= 0.95 \times 0.25 \times 1\\
      &= 0.2375
    \end{align*}
    
    写不命中
    \begin{align*}
      w_{\mathrm{miss}} &= 0.05 \times 0.25 \times 3\\
      &= 0.0375
    \end{align*}

    故带宽占用比例为
    \begin{align*}
      \eta &= r_{\mathrm{hit}} + r_{\mathrm{miss}} + w_{\mathrm{hit}}
      + w_{\mathrm{miss}}\\
      &= 0.35
    \end{align*}
  \item 对写回cache,

    读命中
    \begin{align*}
      r_{\mathrm{hit}}' &= 0.95 \times 0.75 \times 0\\
      &= 0
    \end{align*}


    读不命中 
    \begin{align*}
      r_{\mathrm{miss}}' &= 0.05 \times 0.75 \times (0.3 \times 2
      \times 2 + 0.7 \times 2)\\
      &= 0.0975
    \end{align*}
    
    写命中 
    \begin{align*}
      w_{\mathrm{hit}} &= 0.95 \times 0.25 \times 0\\
      &= 0
    \end{align*}
    
    写不命中
    \begin{align*}
      w_{\mathrm{miss}} &= 0.05 \times 0.25 \times (0.3 \times 2
      \times 2 + 0.7 \times 2)\\
      &= 0.0325
    \end{align*}

    故带宽占用比例为
    \begin{align*}
      \eta &= r_{\mathrm{hit}} + r_{\mathrm{miss}} + w_{\mathrm{hit}}
      + w_{\mathrm{miss}}\\
      &= 0.13
    \end{align*}

  \end{enumerate}
\end{solve}

\input{../footer.tex}

%%% Local Variables: 
%%% mode: latex
%%% TeX-master: t
%%% End: 
